\documentclass{article}

% Symbols
\usepackage{amsmath}
\usepackage{amssymb}

% Language
% \usepackage[spanish]{babel}

% Float managment
\usepackage{float}

% Color
\usepackage{xcolor}

\definecolor{dkgreen}{rgb}{0,0.6,0}
\definecolor{gray}{rgb}{0.5,0.5,0.5}
\definecolor{mauve}{rgb}{0.58,0,0.82}
\definecolor{verylightgray}{rgb}{0.92,0.92,0.92}

% Links
\usepackage{hyperref}
\hypersetup{
  colorlinks,
  urlcolor={mauve}
}

% Code-like text
\usepackage{listings}
\lstset{
  language=bash,
  aboveskip=3mm,
  belowskip=3mm,
  showstringspaces=false,
  columns=flexible,
  basicstyle={\small\ttfamily},
  numbers=none,
  extendedchars=true,
  numberstyle=\tiny\color{gray},
  keywordstyle=\color{blue},
  commentstyle=\color{dkgreen},
  stringstyle=\color{mauve},
  breaklines=true,
  breakatwhitespace=true,
  tabsize=3,
  backgroundcolor = \color{verylightgray},
}


% Margins
\usepackage{geometry}
\addtolength{\hoffset}{-0.5cm}
\addtolength{\textwidth}{1cm}
\addtolength{\voffset}{-0.5cm}
\addtolength{\textheight}{1.9cm}
\addtolength{\headsep}{0.5cm}

% Graphics
\usepackage{graphicx}

% Drawings
\usepackage{tikz}
\usetikzlibrary{arrows,shapes,matrix,decorations.pathmorphing,
                shapes.geometric,calc,babel}


% Header-Footer
\usepackage{fancyhdr}
\pagestyle{fancyplain}
\lhead{Alan Ernesto Arteaga V\'azquez \\
       Mauricio Carrasco Ruiz \\
       C\'esar Hern\'andez Cruz}
\chead{Redes de Computadoras \\ Proyecto 1}
\rhead{Fecha de entrega: \\ 25 de noviembre de 2020}

\renewcommand\headrulewidth{1.5pt}
\makeatletter
\def\headrule{
{\if@fancyplain\let\headrulewidth\plainheadrulewidth\fi
\hrule\@height\headrulewidth\@width\headwidth
\vskip 2pt% 2pt between lines
\hrule\@height.5pt\@width\headwidth% lower line w/.5pt line width
\vskip-\headrulewidth
\vskip-1.5pt}}
\makeatother

% Macros
\newcommand{\ttt}[1]{%
\texttt{#1}%
}


\begin{document}

\section{Registros DNS}

Para realizar esta configuraci\'on, fue necesario
actualizar los name servers de nuestro dominio,
para que usara los name servers de Cloudflare,
esto se configur\'o en el panel de control de
\href{https://get.tech}{get.tech}, bajo la opci\'on
``Name Servers''.   La configuraci\'on puede
verse en la Figura \ref{fig:nameServers}.
\begin{figure}[H]
  \centering
  \includegraphics[width=0.7\textwidth]{DNS/nameServers}
  \caption{Configuraci\'on de los name servers para
           nuestro dominio.}
  \label{fig:nameServers}
\end{figure}

Se configuraron los siguiente registros DNS
externos en Cloudflare.
\begin{enumerate}
  \item A (Web Server - octogatos.tech - 52.87.39.6).

  \item A (Email Server - mail.octogatos.tech - 54.162.84.2).

  \item A (Email Server - postfixadmin.mail.octogatos.tech
                        - 54.162.84.2).

  \item AAAA (Web Server - octogatos.tech
                         - 2600:1f18:22c6:b16:1e05:b4d5:f19:61d).

  \item AAAA (Email Server - mail.octogatos.tech
                           - 2600:1f18:22c6:b00:9ca8:5c2a:c3c:d17f).

  \item CNAME (Web Server - www.octogatos.tech).

  \item MX (Web Server - 10 mail.octogatos.tech).

  \item TXT (Email Server - SPF).

  \item TXT (Email Server - DKIM).

  \item TXT (Email Server - DMARC).
\end{enumerate}
En la Figura \ref{fig:dnsExterno} se muestra la
evidencia respecto a esta configuraci\'on, misma
que puede ser verificada con el comando \ttt{dig},
salvo por aquellos registros para los que el proxy
de Cloudflare est\'a activo.
\begin{figure}[H]
  \centering
  \includegraphics[width=\textwidth]{DNS/dnsExterno}
  \caption{Configuraci\'on de los registros DNS externos.}
  \label{fig:dnsExterno}
\end{figure}

Adicionalmente, se configur\'o el registro PTR para
el servidor de correo, mediante la solicitud a Amazon
para abrir el puerto 25.

Usando Route 53, se configuraron los siguientes
registros.
\begin{enumerate}
  \item A (Web Server - octogatos.tech - 172.31.89.224).

  \item A (Email Server - mail.octogatos.tech - 172.31.32.123).

  \item A (App Server - 172.31.4.254).

  \item A (Data Server - 172.31.4.254).

  \item CNAME (Web Server - www.octogatos.tech).

  \item MX (Web Server - 10 172.31.32.123).
\end{enumerate}
La evidencia de esta configuraci\'on se muestra en
la Figura \ref{fig:createdRecord}
\begin{figure}[H]
  \centering
  \includegraphics[width=0.8\textwidth]{DNS/createdRecord}
  \caption{Configuraci\'on de los registros DNS internos.}
  \label{fig:createdRecord}
\end{figure}

\section{Configuraci\'on NAT}

Como se mencion\'o con anterioridad, con la finalidad
de ahorrar cr\'editos de AWS, configuramos una NAT
independiente de nuestro proyecto, \'unicamente con
dos subredes, una p\'ublica y una privada, cada una
con una instancia de EC2.   A continuaci\'on describimos
c\'omo se llev\'o a cabo dicha configuraci\'on; el punto
principal es que la instancia que corre en la subred
privada no tiene una direcci\'on IP p\'ublica, por lo
que no puede ser accedida desde el exterior de la VPC.
Para conectarnos a \'esta, utilizamos su direcci\'on
IP privada desde la instancia que se encuentra corriendo
en la subred p\'ublica.

Como prerrequisitos, consideramos que ya existe una
instancia corriendo (la que fue creada en la Pr\'actica
2), y que est\'a en la subred por omisi\'on, con la
VPC por omisi\'on, y la Route Table por omisi\'on.
La configuraci\'on de la subred por omisi\'on, que se
utilizar\'a como subred p\'ublica, se muestra en la
Figura \ref{fig:NAT-pubSubnet}.   Adem\'as de la
explicaci\'on brindada por el profesor en Discord,
nos apoyamos en el siguiente documento de AWS.
\href{https://docs.aws.amazon.com/vpc/latest/userguide/VPC_Scenario2.html}{https://docs.aws.amazon.com/vpc/latest/userguide/VPC\_Scenario2.html}

\begin{figure}[H]
  \centering
  \includegraphics[width=\textwidth]{SSNAT/publicSubnet}
  \caption{Configuraci\'on de la subred p\'ublica.}
  \label{fig:NAT-pubSubnet}
\end{figure}

\begin{enumerate}
  \item Creamos una nueva subred, en la misma zona de
    disponibilidad que la red que ya ten\'iamos, y con
    conjunto de direcciones IPv4 en el mismo segmento
    que la anterior.   En la captura de pantalla hay un
    error, hab\'iamos nombrado ``P\'ublica'' a esta red,
    pero esta realmente es la privada (el error se
    corrigi\'o posteriormente).    La configuraci\'on
    puede verse en la Figura \ref{fig:NAT-priSubnet}. Una
    observaci\'on importante es que se deshabilit\'o
    la opci\'on de asignar autom\'aticamente una
    direcci\'on IPv4 a las instancias que se creen
    dentro de esta subred.   Adem\'as de la explicaci\'on
    compartida por el profesor, nos apoyamos en el
    siguiente documento de AWS: \href{https://docs.aws.amazon.com/cloudhsm/latest/userguide/create-subnets.html}{https://docs.aws.amazon.com/cloudhsm/latest/userguide/create-subnets.html}.
    \begin{figure}[H]
      \centering
      \includegraphics[width=0.8\textwidth]{SSNAT/privateSubnet}
      \caption{Configuraci\'on de la subred privada.}
      \label{fig:NAT-priSubnet}
    \end{figure}

  \item Posteriormente, creamos una nueva instancia,
    en la misma \'area de disponibilidad que la instancia
    anterior, pero en la subrede privada reci\'en creada.
    La configuraci\'on de esta instancia puede verse en
    la Figura \ref{fig:NAT-priInstance}.   Es importante hacer
    notar que esta instancia no cuenta con una direcci\'on
    IPv4.
    \begin{figure}[H]
      \centering
      \includegraphics[width=\textwidth]{SSNAT/privateInstance}
      \caption{Configuraci\'on de la instancia privada.}
      \label{fig:NAT-priInstance}
    \end{figure}

  \item El siguiente paso fue crear el NAT gateway.   La
    configuraci\'on resultante aparece en la Figura
    \ref{fig:NAT-nat}.  Cabe resaltar la importancia de
    asignar una IP el\'astica al NAT gateway al momento
    de crearlo. Nos apoyamos en las instrucciones que el
    profesor comparti\'o en Discord y el siguiente
    documento de AWS:
    \href{https://aws.amazon.com/premiumsupport/knowledge-center/nat-gateway-vpc-private-subnet/}{https://aws.amazon.com/premiumsupport/knowledge-center/nat-gateway-vpc-private-subnet/}
    \begin{figure}[H]
      \centering
      \includegraphics[width=\textwidth]{SSNAT/nat}
      \caption{Configuraci\'on del NAT gateway.}
      \label{fig:NAT-nat}
    \end{figure}

  \item Posteriormente, creamos una nueva tabla de
    ruteo.   Es necesario agregar una ruta, con destino
    \ttt{0.0.0.0/0} y target el NAT gateway reci\'en
    creado.   La configuraci\'on resultante puede
    verse en la Figura \ref{fig:NAT-routeTable}.
    \begin{figure}[H]
      \centering
      \includegraphics[width=\textwidth]{SSNAT/routeTable}
      \caption{Configuraci\'on de la nueva tabla de ruteo.}
      \label{fig:NAT-routeTable}
    \end{figure}

  \item Para terminar, asociamos la subred privada con
    la tabla de ruteo creada en el paso anterior. El
    resultado de esa asociaci\'on se observa en la Figura
    \ref{fig:subnetRouteTable}.
    \begin{figure}[H]
      \centering
      \includegraphics[width=\textwidth]{SSNAT/subnetRouteTable}
      \caption{Subred privada asociada a la nueva tabla
      de ruteo.}
      \label{fig:subnetRouteTable}
    \end{figure}
\end{enumerate}

Una vez terminada la configuraci\'on, podemos verificar
que funcione correctamente con los siguientes pasos.

\begin{enumerate}
  \item Ingresar a la instancia p\'ublica, con el
    comando
\begin{lstlisting}
$ ssh -i "practica2.pem" ubuntu@ec2-34-201-69-182.compute-1.amazonaws.com
\end{lstlisting}

  \item Una vez dentro de la instancia p\'ublica, ingresar
    a la privada mediante su direcci\'on IP privada, con
    el comando
\begin{lstlisting}
$ ssh -i "practica2.pem" ubuntu@172.31.0.253
\end{lstlisting}

  \item Verificar la conectividad a internet, por
    ejemplo, con un ping a google.com
\begin{lstlisting}
$ ping google.com
\end{lstlisting}

  \item Al no contar con direcci\'on IP p\'ublica
    ni siquiera es posible intentar conectarnos a
    la instancia privada externamente, pues su
    direcci\'on IP privada es relativa a la subred
    en la que est\'a.
\end{enumerate}

El resultado de los pasos antes descritos puede
revisarse en la Figura \ref{fig:NAT-result}.
\begin{figure}[H]
  \centering
  \includegraphics[width=0.35\textwidth]{SSNAT/inception}
  \caption{It's like inception!}
\end{figure}

\begin{figure}[H]
  \centering
  \includegraphics[width=\textwidth]{SSNAT/result}
  \caption{Subred privada asociada a la nueva tabla
  de ruteo.}
  \label{fig:NAT-result}
\end{figure}


\end{document}
